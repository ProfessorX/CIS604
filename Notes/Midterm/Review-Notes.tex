
%%% Local Variables: 
%%% mode: latex
%%% TeX-master: t
%%% End: 
\documentclass[twocolumn]{article}

% Packages
% \usepackage{fancyhdr}
\usepackage{amsmath}
\usepackage{amssymb}
\usepackage[margin=10mm]{geometry}
\usepackage{xcolor}
\usepackage{graphicx}



% Housekeeping
\pagestyle{empty}


\begin{document}
\section{Artificial Intelligence}
\label{sec:artif-intell}

%%% Ideas:
% There should be more than these note points. Add some when you do
% the second time reviewing.
\begin{itemize}
% \item Life is fucking awesome in the United Arab Emirates!
% \item Life is a bitch. Fuck it or leave it, choose one.
\item An \textbf{agent} is an entity that can perceive and act. This
  course is about designing rational agents. 
\item Rational behavior: doing the right thing. 
\item Environment Types: Fully observable; Deterministic; Episodic;
  Static, Discrete; Single-agent. The counter part: partially
  observable; stochastic; sequential; dynamic; continuous;
  multi-agent. 
\item An agent is anything that can be viewed as perceiving its
  environment through sensors and acting upon that environment through
  actuators.
\end{itemize}


\section{Problem Solving}
\label{sec:problem-solving}

\begin{itemize}
% \item Life is fucking awesome in the United Arab Emirates!
\item A search problem consists of 
  \begin{itemize}
  \item a state space
  \item a successor function (namely \textbf{update function} in data
    mining algorithm series)
  \item a start state (\textbf{initial value}), goal test
    (\textbf{terminating value}) and path cost function (\textbf{we
      say weights in Graph Theory})
  \end{itemize}
\item Problems are often modelled as a state space, a set of states
  that a problem can be in. The set of states forms a graph where two
  states are \textbf{connected} if there is an operation that can be
  performed to transform the first state into the second.
\item A solution is a sequence of actions (a plan) which transforms
  the start state to a goal state.
\item \textbf{State space graph}: A mathematical representation of a
  search problem.
\item \textbf{Search Trees} 
  \begin{itemize}
  \item This is a “what if” tree of plans and outcomes
  \item For most problems, we can never actually build the whole tree
  \end{itemize}
\item \textbf{General Tree Search} Frontier; Expansion; Exploration
  Strategy.
\item \textbf{States vs. Nodes} Nodes in state space graphs are
  problem states; Nodes in search trees are plans. {\color{red}The same problem
  state may be achieved by multiple search tree nodes.}
\item \textbf{Graph Search} Graph Search still  produces a search
  tree; Graph search is almost always better than tree search.
\item 
\end{itemize}

\end{document}
